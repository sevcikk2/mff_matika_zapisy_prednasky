\documentclass{article}
\usepackage[utf8]{inputenc}		% For UTF-8 encoding
\usepackage[IL2]{fontenc}		% Czech font encoding
\usepackage[czech]{babel}		% Czech language support
\usepackage{amsmath}			% For advanced math typesetting
\usepackage{amsfonts}			% For additional fonts
\usepackage{amssymb}			% For additional symbols
\usepackage{amsthm}				% For theorem environments
\usepackage{geometry}			% To adjust page margins
\usepackage{hyperref}			% For clickable links and references
\usepackage[dvipsnames]{xcolor}	% For colors - highliting
\usepackage{soul}				% For command hl = highlight
\usepackage{enumerate}			% For lists
\usepackage{soulpos}

\DeclareRobustCommand{\hldef}[1]{{\sethlcolor{SkyBlue}\hl{#1}}}
\DeclareRobustCommand{\hldefmath}[1]{{\colorbox{SkyBlue}{#1}}}
\DeclareRobustCommand{\hlveta}[1]{{\sethlcolor{magenta}\hl{#1}}}
\DeclareRobustCommand{\hlvetamath}[1]{{\colorbox{magenta}{#1}}}

% Set page margins
\geometry{a4paper, margin=1in}

% Define theorem environment
\newtheorem{vetat}{Věta T}[section]
\newtheorem{vetal}[vetat]{Věta L}
\newtheorem*{tvrzeni}{Tvrzení}
\newtheorem*{lemma}{Lemma}
\newtheorem*{dusledek}{Důsledek}
\newtheorem*{pozn}{Poznámka}
\newtheorem*{pozorovani}{Pozorování}
\theoremstyle{definition}
\newtheorem*{definice}{Definice}

\title{Matematická analýza}
\author{Kateřina Ševčíková}
\date{Poslední úprava: \today}

\begin{document}

\maketitle

\tableofcontents
\newpage

\section{Úvod}
Logika, důkazy, mohutnost množin.

\section{Posloupnosti}
Limity posloupností.

\section{Funkce jedné reálné proměnné – limita a spojitost}
Limity funkcí.

\section{Funkce jedné reálné proměnné – derivace a Taylorův polynom}
Derivace a Taylor.

\section{Řady}
Konvergence řad.

\section{Primitivní funkce}
Primitivní funkce, integrace.

\section{Určitý integrál}
Riemannův a Newtonův integrál.

\section{Obyčejné diferenciální rovnice}
= diferenciální rovnice s jednou proměnnou, s více proměnnými jsou to parciální diferenciální rovnice

\subsection{Řešení, existence a jednoznačnost}
\(y'(x) = f(x, y(x))\) má řešení, je-li f hezká

\begin{definice}
	Nechť \(\Phi : \Omega \subset \mathbb{R}^{n+2} \to \mathbb{R}\).\ \hldef{Obyčejnou diferenciální rovnicí} (zkratka ODR) 
	n-tého řádu nazveme 
\begin{equation}
	\Phi(x, y(x), y'(x), \ldots, y^{(n)}(x)) = 0
\end{equation}
\end{definice}

\begin{definice}
	\hldef{Řešení} obyčejné diferenciální rovnice ma otevřeném intervalu $I \subset \mathbb{R}$ je funkce splňující
\begin{enumerate}[(i)]
	\item existuje $y^{(k)}(x)$ vlastní pro $k = 1, 2, \ldots, n$ pro všechna $x \in I$ 
	\item rovnice (1) platí pro všechna $x \in I$
\end{enumerate}
	Řešením je dvojice $(y, I)$.
\end{definice}

\begin{definice}
	Řekneme, že $(\tilde{y}, \tilde{I})$ je \hldef{rozšířením} $(y, I)$, pokud
\begin{enumerate}[(i)]
	\item $\tilde{y}$ je řešení (1) na $\tilde{I}$
	\item $I \subset \tilde{I}$
	\item $y = \tilde{y}$ na $I$
\end{enumerate}
	Řekneme, že $(y, I)$, je \hldef{maximální řešení}, pokud nemá rozšíření.
\end{definice}

\begin{definice}
	Řekneme, že $I \subset \mathbb{R}^{n}$ je \hldef{otevřený interval}, pokud existují otevřené
	intervaly $I_1, I_2, \ldots, I_n$ tak, že $I = I_1 \times \cdots \times I_n$.
\end{definice}

\begin{definice}
	Nechť $c \in \mathbb{R}^{n}$ a $r>0$. Definujeme \hldef{otevřenou kouli} jako 
	\[ B(c, r) = \left\{ x \in \mathbb{R}^{n} : |x - c|  
	= \sqrt{\sum_{i = 1}^{n} (x_i - c_i)^{2}} > r \right\} \]
\end{definice}

\begin{definice}
	Nechť $I \subset \mathbb{R}^{n}$ je otevřený interval a $f: I \to \mathbb{R}$ je funkce. Řekneme, že $f$ je
	\hldef{spojitá} v bodě $x_0 \in I$, pokud $\forall \epsilon > 0 \  \exists \delta > 0 \ 
	\forall x \in B(x_0, \delta) \cap I$ platí $|f(x) - f(x_0)| < \epsilon$.
	Řekneme, že $f$ je spojitá na $I$, pokud je spojitá ve všech bodech I.
\end{definice}

\begin{pozorovani}
	...
\end{pozorovani}

\begin{proof}
	Důkaz pozorování.
\end{proof}

\begin{dusledek}
	\(P(x,y)\) polynom dvou proměnných je spojitá funkce na \(\mathbb{R}^{2}\)
\end{dusledek}

\begin{vetat}
	\hlvetamath{Peano s $y^{(n)}$}
	\\
	Nechť \(I \subset \mathbb{R}^{n+1}\) je otevřený interrval, \(f: I \to \mathbb{R}\) je spojitá, a \linebreak[1]
	\([x_0, y_0, \dots, y_{n-1}] \in I\). Pak \linebreak[1] \(\exists \delta > 0\) a v okolí \linebreak[1]
	\(x_0\) existuje interval \linebreak[1] \((x_0- \delta , x_0 + \delta)\) a funkce \(y(x)\) definována na 
	\linebreak[1] \((x_0 - \delta, x_0 + \delta)\) tak, že \(y(x)\) splňuje ODR \linebreak[1]
	\(y^{(n)}(x) = f(x, y(x^), \dots, y^{(n-1)}) \forall x \in (x_0 - \delta, x_0 + \delta)\) s počáteční podmínkou 
	\(y(x_0) = y_0, y'(x_0) = y_1, \dots , y^{(n-1)}(x_0) = y_{n-1}\).
\end{vetat}

\begin{vetal}
POkus
\end{vetal}

\end{document}

